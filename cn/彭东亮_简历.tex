%https://tex.stackexchange.com/questions/448129/arara-automation-tool-script-not-found
%the following content of setting arara must stay before \documentclass[]{}
% arara: xelatex
% arara: biber
% arara: xelatex
% arara: xelatex

%\listfiles

% !TEX encoding = UTF-8 Unicode
% !Mode:: "TeX:UTF-8"
% !TEX TS-program = xelatex
\documentclass{resume}

%\usepackage{zh_CN-Adobefonts_external} % Simplified Chinese Support using external fonts (./fonts/zh_CN-Adobe/)
%\usepackage{zh_CN-Adobefonts_internal} % Simplified Chinese Support using system fonts
\usepackage{linespacing_fix} % disable extra space before next section
%\usepackage{cite}

\usepackage{xspace}
\newcommand{\Astar}{A$^{\!\star}$\xspace}
%\usepackage[nobottomtitles*]{titlesec} %keep a \section from being at the end 
%%of a page
\usepackage[defaultlines=3,all]{nowidow}

%\usepackage{booktabs,siunitx} %for \toprule, \midrule, \bottomrule
\usepackage{array}
\usepackage{tabularx}

\usepackage{xeCJK}
\usepackage{CTEX} % to display special character like ‘旸’ or '璆'


\newcolumntype{s}{>{\hsize=0.3\hsize\centering\arraybackslash}X}
\newcolumntype{Y}{>{\centering\arraybackslash}X}

%https://tex.stackexchange.com/questions/823/
\usepackage{hyperref}
\usepackage{xcolor}
\hypersetup{
    colorlinks,
    linkcolor={red!50!black},
    citecolor={blue!50!black},
    urlcolor={blue!80!black},
%    hidelinks,
}



\usepackage[
%backend=biber,
%style=numeric,
style=numeric-comp, %number the references, compact
%style=authoryear, %no indices before names; year after names
%style=alphabetic, %style for my thesis
%
%sorting=ydnt, %sort in bibliography by name, year, volume, title
sorting=none,
sortcites, %sort in content by numbering
hyperref=true,%transform citations and back references into 
%clickable hyperlinks
backref=true, %to print back references in the bibliography
backrefstyle=three, %Compress three or more consecutive pages 
%to a range
%,
%citestyle=authoryear,
maxbibnames=10, %the author names appearing in bibliography
maxcitenames=2, %the author names appearing in text
%giveninits=true,
%terseinits=true,%remove periods after initials of first names
isbn=false,
%doi=false,url=false,
eprint=false,
%dashed=false,
%useprefix=false,%sort bibliography use prefix like "van"
natbib=true,
%style=ACM-Reference-Format,
language=american,
%abbreviate=true, 
dateabbrev=true,
defernumbers=true,
refsection=section
]{biblatex}
\addbibresource{Reference/BibReference.bib}

%if a reference has doi, then we print only doi in bibliography;
%otherwise, we print only url
%we don't use eprint
%we must deactivate eprint by using "eprint=false"
\renewbibmacro*{doi+eprint+url}{% 
	\iftoggle{bbx:url} 
	{\iffieldundef{doi}{\usebibmacro{url+urldate}}{}} 
	{}% 
	%	\newunit\newblock 
	%	\iftoggle{bbx:eprint} 
	%	{\usebibmacro{eprint}} 
	%	{}% 
	\newunit\newblock 
	\iftoggle{bbx:doi} 
	{\printfield{doi}} 
	{}}	





%remove the information we don't want to present
\AtEveryBibitem{% Clean up the bibtex rather than editing it
	\clearlist{address}
	\clearfield{date}
%	\clearfield{eprint}
	\clearfield{isbn}
	\clearfield{issn}
	\clearlist{location}
	\clearfield{month}
	\clearlist{language}
	%	\clearfield{series}
	
	\ifentrytype{book}{}{% Remove publisher and editor except for books
		\clearlist{publisher}
		%		\clearname{editor}
	}
}

%For the references, the following commands make the titles 
%lowercase (sentence case),
%while keep book/journal/conference etc. names unchanged 
\DeclareFieldFormat{sentencecase}{\MakeSentenceCase{#1}}
\renewbibmacro*{title}{%
	\ifthenelse{\iffieldundef{title}\AND\iffieldundef{subtitle}}
	{}
	{\ifthenelse{\ifentrytype{article}\OR\ifentrytype{inbook}%
			\OR\ifentrytype{incollection}\OR\ifentrytype{inproceedings}%
			\OR\ifentrytype{inreference}}
		{\printtext[title]{%
				\printfield[sentencecase]{title}%
				\setunit{\subtitlepunct}%
				\printfield[sentencecase]{subtitle}}}%
		{\printtext[title]{%
				\printfield[titlecase]{title}%
				\setunit{\subtitlepunct}%
				\printfield[titlecase]{subtitle}}}%
		\newunit}%
	\printfield{titleaddon}}

%\DeclareNameAlias{author}{family-given}%lastname before firstname



%\item \datedline{
%    \begin{minipage}[t]{0.75\textwidth}
%     Supervising Konrad Jarocki (as second supervisor),
%          Parallel step assignment for continous generalization.
%    \end{minipage}}{2019/09--2020/07}

\newcommand{\itemdatedminipage}[2]
    {\item\datedline{\begin{minipage}[t]{0.75\textwidth}#1\end{minipage}}{#2}\vspace{0.2ex}}
\newcommand{\itemdatedminipageX}[2]
    {\item\datedline{\begin{minipage}[t]{0.75\textwidth}#1\end{minipage}}{#2}\vspace{1.5ex}}



\begin{document}
%\pagenumbering{gobble} % suppress displaying page number

\name{彭东亮}

%{d.l.peng@tudelft.nl}{mobilephone}{homepage}
% be careful of _ in emaill address
\contactInfo{15873168540}{微信号:pengdlzn}{pengdlzn@qq.com}{博士后}
% {E-mail}{mobilephone}
% keep the last empty braces!
%\contactInfo{xxx@yuanbin.me}{(+86) 131-221-87xxx}{}
 
\section{基本情况}
本人出生于1987年10月26日,湖南浏阳。
本人工作负责,勤勉上进,适应力强,乐观向上,易于相处。
学习了多个专业,包括测绘工程(本科),地图制图学与地理信息工程(硕士),计算机科学(博士)。
本科,硕士,博士毕业论文都获得“优”的成绩。
研究方向包括地理信息系统算法,基于最优化的地图制图,地图多尺度表达,地图连续综合,带有平滑渐变的网络地图。
发表了多篇论文,并多次参加会议并做口头报告。
掌握了众多算法,包括A*、Dijkstra、动态规划、线性规划、整数规划(运筹优化)、最小二乘校正、最小生成树、马尔可夫链。
掌握了多门编程语言,包括Python、PostgreSQL、PostGIS、JavaScript、HTML5、CSS3、WebGL、C\#、Visual Basic .NET。
曾获得中南大学优秀研究生和比亚迪奖学金优秀学生奖等奖项。
参与组织了两个学术活动。给多个期刊和会议审稿。
给学生上课并指导帮助了多个学生完成硕士和本科毕业论文。
个人网站为\url{www.researchgate.net/profile/Dongliang_Peng}。不当之处请多多指教,非常感谢!\footnotemark
\footnotetext{本简历更新于\today。}

% \section{\faGraduationCap\ 教育背景}
\section{工作经历}
\datedsubsection{\textbf{博士后},地理信息系统技术,
代尔夫特理工大学,荷兰}
{2018/05--2020/12}
\begin{tabular}{ll}	
	课题:  &  任意比例尺网络地图 (Vario-scale web maps) \\
	导师:      & Peter van Oosterom 和 Martijn Meijers
\end{tabular}%


% \section{\faGraduationCap\ 教育背景}
\section{教育背景}
\datedsubsection{\textbf{博士},计算机科学,维尔茨堡大学,德国}
{2012/10--2017/12}
\begin{tabular}{ll}			
	论文:  & 基于最优化的地图连续综合方法 \\
           & (An Optimization-Based Approach for Continuous Map Generalization)\footnotemark\\
	成绩:      & magna cum laude (优)\\
	导师:      & Alexander Wolff 和 Jan-Henrik Haunert
\end{tabular}%
\footnotetext{博士论文开放获取网址为
    \url{https://doi.org/10.25972/WUP-978-3-95826-105-1}。}
%
\datedsubsection{\textbf{硕士},地图制图学与地理信息工程,中南大学,中国}
{2009/09--2012/05}
\begin{tabular}{ll}	
	论文:  & 面向地图连续综合的线状要素Morphing变换方法研究\footnotemark\\
	成绩:      & 中南大学优秀硕士学位论文\\
	导师:      & 邓敏
\end{tabular}
\footnotetext{硕士论文网址为 \url{http://cdmd.cnki.com.cn/Article/CDMD-10533-1012478478.htm}。}
%
\datedsubsection{\textbf{学士},测绘工程,中南大学,中国}
{2005/09--2009/06}
\begin{tabular}{ll}	
	论文:  & 基于AutoCAD的矢量数据更新方法\\
	成绩:      & 优\\
	导师:      & 邓敏
\end{tabular}



% \section{\faCogs\ IT 技能}
\section{研究兴趣}
% increase linespacing [parsep=0.5ex]
\begin{itemize}[parsep=0ex]
\item 地理信息系统算法,基于最优化的地图制图,地图多尺度表达, 
    地图连续综合,带有平滑渐变的网络地图
\end{itemize}

\section{获奖和荣誉}
\begin{itemize}[parsep=0ex]
\itemdatedminipageX{地理信息科技进步一等奖(序11),
    ``多源多尺度空间数据不一致性探测处理的理论与方法''。}{2013/09}
\itemdatedminipage{中南大学优秀研究生。}{2011/12}
\itemdatedminipage{比亚迪奖学金优秀学生奖。}{2011/12}
\itemdatedminipageX{\textbf{彭东亮},邓敏,赵彬彬,``基于Morphing的河网多尺度变换方法研究'',
    2011年地理信息产业``苍穹杯''青年优秀论文三等奖。}{2011/10}
\itemdatedminipageX{刘启亮,邓敏,\textbf{彭东亮},徐震,``基于场论的空间聚类有效性评价方法研究'',
    ``中测新图杯''青年优秀论文一等奖。}{2009/10}
\end{itemize}


% \section{\faGraduationCap\ 教育背景}
\section{项目经历}
\datedsubsection{比较任意比例尺地图与多比例尺地图}
{2020/01--2020/12}
\begin{tabular}{lp{15cm}}			
	本人角色:  & 科研人员 \\
	主要业绩:  & 任意比例尺地图是地图行业的一个重要发展方向。
                本项目通过用户测试,定量分析任意比例尺地图相较于多比例尺地图帮助用户提高的使用效率。
                本人在已有工具的基础上开发了两种形式的网络地图:任意比例尺地图和多比例尺地图。
                我们的合作者在此基础上做用户对比测试。
                发表了两篇相关论文:
                1、Paralleling generalization operations to support smooth zooming: 
                case study of merging area objects;
                2、Multi-layer vario-scale web map comparer with dynamic transitions and 
                visual analytical tool。
\end{tabular}%
%
\datedsubsection{空间异常多尺度探测方法研究}
{2010/06--2011/12}
\begin{tabular}{lp{15cm}}	
	本人角色:  & 主持 \\
	主要业绩:  & 通过本人与项目组成员的合作研究,主要取得了以下成绩。
                发表论文两篇:
                1、一种基于场论的层次空间聚类算法,
                2、一种基于力学思想的空间聚类有效性评价。
                编写可视化系统软件包一个:空间异常探测软件(EasyDetector)。
\end{tabular}
%
\datedsubsection{地图更新中多尺度空间目标匹配的层次理论与方法}
{2009/01--2011/12}
\begin{tabular}{lp{15cm}}	
	本人角色:  & 科研人员 \\
	主要业绩:  & 通过本人与项目组成员的合作研究,主要取得了以下成绩。
                发表论文四篇:
                1、一种基于弯曲结构的线状要素Morphing方法,
                2、Multi-Scale Transformation of River Network based on the Morphing Technology,
                3、基于Morphing的河网多尺度变换方法研究,
                4、顾及BLG树结构特征的线状要素Morphing变换方法。
                编写可视化系统软件包两个:
                1、时空插值分析软件(EasyInterpolator),
                2、空间聚类分析软件(EasyCluster)。                
\end{tabular}
%
\datedsubsection{测绘农田地图}
{2008/07--2008/08}
\begin{tabular}{lp{15cm}}	
	本人角色:  & 测绘员 \\
	主要业绩:  & 使用全站仪采集农田边界数据,然后把数据导入到AutoCAD中,进一步编辑生成农田地图。                
\end{tabular}

\section{论文}

%papers appearing in this list will also appear in the list of all publications
\nocite{Peng2020AreaAgg,Peng2020Parallel,Peng2016Admin,Peng2017Building,Deng2015}
\printbibliography[title=代表论文,
    keyword={keypub},heading=subbibliography,resetnumbers]


%\nocite{
%%first
%Peng2020Viewer,Peng2017AStar,Peng2014DataStr,Peng2014Sufficiently,
%Peng2013LSA,Peng2013Law,Peng2013Similarity,
%Peng2012BLG,Peng2012RiverEn,Peng2012River,
%Peng2011Bend,
%%second
%Deng2012Bend,Peng2011Cluster,
%Zhao2016Assimilation,
%%third
%Zhao2016Inconsistencies,
%Liu2011Validity,
%%other
%Meijers2018Framework,
%Wang2015Change,
%Liu2011Multi,Zhao2010Network}
%
%%Peng2020AreaAgg,Peng2016Admin,Deng2015,Peng2017Building,Peng2020ParallelICA,
%
%
%%\begin{refcontext}[sorting=ydnt]{} % sort chronologically
%\begin{refcontext}[sorting=none]{} % sort chronologically
%\printbibliography[title=其它论文,
%    keyword={otherpub},heading=subbibliography] 
%\end{refcontext}


\section{学术交流}
\subsection{会议}
\begin{itemize}[parsep=0ex]
\itemdatedminipageX{23rd International Cartographic 
    Association	Workshop on Generalisation and Multiple 
    Representation (ICAGM'20); \textbf{oral presentation}.
    Delft, The Netherlands.}{11/05--11/06, 2020}
\itemdatedminipageX{ISPRS TC IV Mid-term Symposium 
	``3D Spatial Information Science---The Engine of Change'' 
	(Volume XLII-4). 
    Delft, The Netherlands.}{10/01--10/05, 2018}
\itemdatedminipageX{25th ACM SIGSPATIAL International Conference on Advances
	in Geographic Information Systems (ACMGIS'17).
    Redondo Beach, California, USA.}{11/07--11/10, 2017}
\itemdatedminipageX{3rd International Workshop on Smart Cities and 
    Urban Analytics (UrbanGIS'17); \textbf{oral presentation}.
    Redondo Beach, California, USA.}{11/07, 2017}
\itemdatedminipageX{28th International Cartographic 
    Conference (ICC'17); \textbf{oral presentation}. 
    Washington DC, USA.}{07/02--07/07, 2017}
\itemdatedminipageX{20th International Cartographic Association 
	Workshop on Generalisation and Multiple Representation (ICAGM'17).
    Washington DC, USA.}{07/01, 2017}
\itemdatedminipageX{19th Association of Geographic 
    Information Laboratories in Europe International Conference on Geographic 
	Information Science (AGILE'16); \textbf{oral presentation}.
    Helsinki, Finland.}{06/14--06/17, 2016}
\itemdatedminipageX{19th International Cartographic Association 
	Workshop on Generalisation and Multiple Representation (ICAGM'16).	
    Helsinki, Finland.}{06/14, 2016}         
\itemdatedminipageX{22nd Annual Conference of 
Geographical Information Systems Research UK (GISRUK'14); \textbf{oral presentation}.
    Glasgow, UK.}{04/16--04/18, 2014}     
\itemdatedminipageX{26th International Cartographic Conference (ICC'13).
    Dresden, Germany.}{08/25--08/30, 2013} 
\itemdatedminipageX{16th International Cartographic 
    Association	Workshop on Generalisation and Multiple 
    Representation (ICAGM'13); \textbf{oral presentation}.
    Dresden, Germany.}{08/23--08/24, 2013}
%https://generalisation.icaci.org/prevevents/76-workshop2013papers.html    
\itemdatedminipageX{The European Workshop on 
    Computational Geometry (EuroCG'13).
    Braunschweig, Germany.}{03/17--03/20, 2013}    
\itemdatedminipage{中国地理信息产业大会。
    中国北京。}{10/25--10/26, 2011}
\end{itemize}

\subsection{论坛}
\begin{itemize}[parsep=0ex]
\itemdatedminipage{NCG symposium. Delft, The Netherlands.}{11/05, 2020}
\itemdatedminipage{Geomatics Day. Delft, The Netherlands.}{06/26, 2020}
\itemdatedminipage{NCG symposium; \textbf{oral presentation}. 
    Enschede, The Netherlands.}{11/21, 2019}
\itemdatedminipage{Geomatics Day. Delft, The Netherlands.}{06/28, 2019}    
\itemdatedminipage{Seminar Geo-Information Systems in Action. 
    Delft, The Netherlands.}{10/19, 2018}
\itemdatedminipage{Geomatics Day. Delft, The Netherlands.}{06/22, 2018}
\itemdatedminipageX{Map generalization and multiple-/vario-scale representations, 
    a seminar to close the STW project Vario-scale Geo-information. 
    Delft, The Netherlands.}{06/12, 2017}
\itemdatedminipageX{3rd PhD Colloquium of the DGK Section on Geoinformatics and the DGPF
    Working Group on Geoinformatics. 
    W\"urzburg, Germany.}{03/07, 2017}
\itemdatedminipageX{2nd PhD Colloquium of the DGK Section on Geoinformatics and the DGPF
    Working Group on Geoinformatics; \textbf{oral presentation}. 
    Bonn, Germany.}{02/23, 2016}
\itemdatedminipage{中南大学第二届测绘研究生学术论坛;\textbf{口头报告}(三等奖)。
    中国长沙。}{12/22--12/23, 2011}
\itemdatedminipage{中南大学第一届测绘研究生学术论坛;\textbf{口头报告}(三等奖)。 
    中国长沙。}{12/12--12/13, 2009}
\end{itemize}

\subsection{短期培训}
\begin{itemize}[parsep=0ex]
\itemdatedminipage{Geometric Algorithms in the Field; \textbf{Poster}. 
    Leiden, The Netherlands.}{06/23--26/27, 2014}
\itemdatedminipage{EuroGIGA Fall School. 
    W\"urzburg, Germany.}{10/08--10/12, 2012} 
\end{itemize}


\subsection{访问}
\begin{itemize}[parsep=0ex]
\itemdatedminipage{Dr.\ Guillaume Touya, French National Mapping Agency (IGN),
    Saint-Mand\'e, France.}{09/12--09/23, 2016}
\itemdatedminipage{Dr.\ Jan-Henrik Haunert, University of Osnabr\"uck,
    Osnabr\"uck, Germany.}{03/09--03/13, 2015}
\itemdatedminipage{Dr.\ Jan-Henrik Haunert, University of Osnabr\"uck,
    Osnabr\"uck, Germany.}{07/28--08/01, 2014}
\end{itemize}


\section{审稿}
\subsection{期刊}
\begin{itemize}[parsep=0ex]
\item 测绘学报
\item Computers and Geosciences (2次)
\item International Journal of Digital Earth
\item International Journal of Geographical Information Science
\item Journal of Spatial Science
\item 武汉大学学报$\cdot$信息科学版 (3次)
\end{itemize}

\subsection{会议}
\begin{itemize}[parsep=0ex]
\item 25th International Symposium on Algorithms and Computation (ISAAC'16)
\item 23rd ICA Workshop on
    Map Generalisation and Multiple Representation (ICAGM'20,3篇)
\end{itemize}

\section{学术活动组织}
\begin{itemize}[parsep=0ex]
\itemdatedminipageX{23rd ICA Workshop on Map Generalisation and Multiple Representation,
    member of Program Committee. Delft, The Netherlands.}{11/05--11/06, 2020}
\itemdatedminipageX{NCG Workshop on Creating Interactive Online maps, \textbf{organizer}. 
 Delft, The Netherlands.\footnotemark}{11/05, 2020}
\footnotetext{The link to the workshop is \url{https://pengdlzn.github.io/events/interactive-online-maps/}.}    
\end{itemize}

\section{教学}
\subsection{正式课程}
\begin{itemize}[parsep=0ex]
\item 面向地球空间信息科学的Python编程 (Python Programming for Geomatics)
\end{itemize}

\subsection{习题课}
\begin{itemize}[parsep=0ex]
\item 地理信息系统算法 (Algorithms for GIS)
\item 计算几何 (Computational Geometry)
\end{itemize}

\subsection{其它}
\begin{itemize}[parsep=0ex]
\item 空间优化 (Spatial Optimization,正在开发)
\item 荷兰大学教师资格证 (University Teaching Qualification,已完成90\,\%)
\end{itemize}

%\item 面向地球空间信息科学的Python编程 Python Programming for Geomatics
%    (Some lectures: 2019~Q1\footnotemark, 2020~Q1; Tutorials: 2018~Q1, 2019~Q1, 2020~Q1)
%    \footnotetext{Q1: quarter 1 of an academic year.}
%\item Algorithms for GIS
%    (Tutorials: 2016~SS\footnotemark, 2017~SS)
%    \footnotetext{SS: summer semester; WS: winter semester.} 
%\item Computational Geometry
%    (Tutorials: 2015~WS, 2016~WS)
%\item Spatial Optimization, under development
%\item Working on Dutch \emph{University Teaching Qualification}, 
%    finished 80\,\%


%\clearpage

\section{指导帮助毕业论文}
\subsection{硕士}
\begin{itemize}[parsep=0ex]
\itemdatedminipageX{作为第二导师指导Charlie Groenewegen,
    ``Locations for low cost large-scale green hydrogen production systems 
    in Europe and North Africa''.}
    {2020/09--}
\itemdatedminipageX{作为第二导师指导Konrad Jarocki,
    ``Parallel step assignment for continuous generalization''.}
    {2019/09--2020/07}
\itemdatedminipageX{帮助Felipe Reinel, 
    ``Multidimensional labor resource visualization 
    for integrated turnarounds''.}
    {2019/01--2019/06}    
\itemdatedminipageX{帮助Yannick Brangers, 
    ``Project A-Locate: Using location-allocation
    modelling to optimise human resources in retail environments''.}
    {2018/09--2019/06}      
\end{itemize}

\subsection{本科}
\begin{itemize}[parsep=0ex]
\itemdatedminipage{帮助王航,
    ``面状要素地图连续综合方法研究''。}
    {2014/03--2014/06}
\itemdatedminipage{帮助谢坤,
    ``以拓扑形变最小为准则的面状要素地图鱼眼视图方法''。}
    {2013/03--2013/06}    
\itemdatedminipage{帮助张琦,
    ``基于ArcEngine的济南水雨情信息系统研究''。}
    {2012/03--2012/06}
\itemdatedminipage{帮助胡敏,
    ``地图综合中基于结构的线状要素Morphing变换方法研究''。}
    {2012/03--2012/06}    
\itemdatedminipage{帮助刘海燕,
    ``空间聚类有效性评价方法对比研究''。}
    {2010/03--2010/06}    
\end{itemize}


\section{软件著作权}
\begin{itemize}[parsep=0ex]
\item 邓敏,\textbf{彭东亮},刘启亮,刘慧敏,彭思岭,徐震,黄雪萍,张朋东,
    ``时空插值分析软件(EasyInterpolator)'',
   证书号:软著登字第0244971号,登记号:2010SR056698, 完成日期: 2010年10月10日。
\item 邓敏,刘启亮,\textbf{彭东亮},刘慧敏,石岩,李光强,王佳璆,梅小明,赵玲,
   ``空间异常探测软件(EasyDetector)'',
   证书号:软著登字第0221873号,登记号:2010SR033600,完成日期: 2010年06月04日。
\item 邓敏,刘启亮,\textbf{彭东亮},李光强,刘慧敏,
   ``空间聚类分析软件(EasyCluster)'',
   证书号:软著登字第0209447号,登记号:2010SR021174,完成日期: 2010年03月10日。   
\end{itemize}

\section{编程语言}
\begin{itemize}[parsep=0ex]
\item 基于Python,PostgreSQL,PostGIS,JavaScript,HTML,CSS,WebGL 
    开发了``任意比例尺网络地图(vario-scale web maps)''。\footnotemark%
    \footnotetext{一个带有并行平滑合并网络地图的示例:
	   \url{https://pengdlzn.github.io/webmaps/2020/10/merge/top10nl-0.01.html}.}
\item 基于C\#及函数库ArcGIS Objects,CPLEX,Eigen,Clipper,Excel
    开发了``地图连续综合软件(ContinuousGeneralizer)''。\footnotemark
    \footnotetext{地图连续综合软件(ContinuousGeneralizer)可以在GitHub上开放获取:	    
	   \url{https://github.com/IGNF/ContinuousGeneralisation}.}    
\item 基于C\#及函数库ArcGIS Objects开发了``时空插值分析软件(EasyInterpolator)''。
\item 基于Visual Basic .NET及函数库MapObjects开发了``空间异常探测软件(EasyDetector)''。
\item 基于Visual Basic .NET及函数库MapObjects开发了``空间聚类分析软件(EasyCluster)''。 
\item 熟悉:Java, C.
\item 有少许经验:C++, XML, Matlab, R.
\end{itemize}


\section{常用专业工具}
\begin{itemize}[parsep=0ex]
\item Apache, ArcMap, ArcGIS Pro, FME, Git, Inkscape, Ipe, 
    LaTeX, ParaView, PostgreSQL, QGIS
\end{itemize}

\section{语言}
\begin{itemize}[parsep=0ex]
\item 中文,母语
\item 英语,B2(依据欧洲共同语言参考标准),测试于2018年9月26日
\item 德语,A2(依据欧洲共同语言参考标准),测试于2014年7月11日
\end{itemize}

\section{爱好}
\begin{itemize}[parsep=0ex]
\item 运动:足球,壁球,滑雪,滑冰,游泳,乒乓球,羽毛球,桌球,电子竞技等
\item 桌游:象棋,围棋,Magic等
\end{itemize}

\section{参考人}
\begin{itemize}[parsep=1ex]

\item Prof.\ Dr.\ \textbf{Peter van Oosterom}.
	Section GIS technology,
	Faculty of Architecture and the Built Environment, 
	Delft University of Technology, The Netherlands.
	Email: P.J.M.vanOosterom@tudelft.nl,
	Homepage: \url{www.gdmc.nl/oosterom/}.

\item Prof.\ Dr.\ \textbf{Alexander Wolff}.
	Chair of Algorithms, Complexity, and Knowledge-Based Systems,
	Faculty of Mathematics and Computer Science, 
	University of W\"urzburg, Germany.
	Email: alexander.wolff@uni-wuerzburg.de,
	Homepage: \url{www.informatik.uni-wuerzburg.de/en/algo/staff/wolff-alexander/}.   

\item Prof.\ Dr.\ \textbf{Jan-Henrik Haunert}. 
	Institute of Geodesy and Geoinformation,
    Faculty of Agriculture,
	University of Bonn, Germany.
	Email: haunert@igg.uni-bonn.de,
	Homepage: \url{www.geoinfo.uni-bonn.de/haunert}.

%\item \textbf{Prof.\ Dr.\ 邓敏}. 
%    中南大学, 地球科学与信息物理学院, 地理信息系.
%	Email: dengmin@csu.edu.cn,
%	Homepage: \url{https://faculty.csu.edu.cn/dengmin_csu/zh_CN/}.     
\end{itemize}
   






%\section{获奖和荣誉}
%\datedsubsection{\textbf{阿里巴巴集团 | Alibaba}, 前端开发工程师}{2017.6-2017.9}
%\begin{itemize}
%%   \item 飞猪北京前端团队全面负责各交通线的票务(机票/火车票/汽车票) web 应用与事业群基础架构研发
%  \item 独立负责车站地图开发(React),通过HTML5 本地存储及JSBridge实现在阿里全系应用中发布上线
%  \item 独立负责BU SPM chrome插件开发,支付成功/订单详情等页面的开发与交叉营销的接入工作
%\end{itemize}
%
%\datedsubsection{\textbf{北京腾云天下科技有限公司 | TalkingData},数据挖掘与可视化工程师}{2015.11-2017.5}
%\begin{itemize}
%  \item \textbf{利用海量用户定位数据,对城市空间及人群移动特征进行研究。}第一个课题是基于香农熵和人群出行模式,构建城市网格与用户矩阵分析城市多样性/流动性分布;可视分析平台前端与可视化基于D3/Vue/Express开发,数据分析与存储采用Python/MySQL/MongoDB技术,为了均衡大数据情况下的页面可视化渲染消耗用canvas替代svg。第二个课题是对海量商场定位数据做人群分类与可视化查询,依据该系统撰写的论文被CIKM 2016(DAVA Workshop)录用,并收录于中科院软件所年会成果集
%  \item 负责数据科学部HQ LAB的可视化原型开发,主导 TalkingMind 平台系统设计与前端开发
%\end{itemize}
%
%\datedsubsection{\textbf{北京格灵深瞳信息技术有限公司 | DeepGlint},Web开发工程师}{2015.7-2015.9}
%\begin{itemize}
%  \item \textbf{独立负责MUSE部门的可视化组件研发。}与平台研发、设计协作完成 DeepGlint Developer 平台可视化图表组件的集成开发,符合完全定制化渲染、响应式布局与实时更新等特点
%  \item 利用 D3+Vue+WebGL(Three.js) 尝试实现三维空间的人群移动可视化
%\end{itemize}

% \begin{onehalfspacing}
% \end{onehalfspacing}

% \datedsubsection{\textbf{DID-ACTE} 荷兰莱顿}{2015年3月 - 2015年6月}
% \role{本科毕业设计}{LIACS 交换生}
% 利用结巴分词对中国古文进行分词与词性标注,用已有领域知识训练形成 classifier 并对结果进行调优
% \begin{onehalfspacing}
% \begin{itemize}
%   \item 利用结巴分词对中国古文进行分词与词性标注
%   \item 利用已有领域知识训练形成 classifier, 并用分词结果进行测试反馈
%   \item 尝试不同规则,对 classifier 进行调优
% \end{itemize}
% \end{onehalfspacing}

%\section{竞赛获奖/项目作品}
%% increase linespacing [parsep=0.5ex]
%\begin{itemize}[parsep=0.2ex]
%%   \item LeetCodeOJ Solutions, \textit{https://github.com/hijiangtao/LeetCodeOJ}
%  \item 第三届中国软件杯大学生软件设计大赛\textbf{全国一等奖}( \textit{http://www.cnsoftbei.com/} ),2014 年8月
%  \item 中国机器人大赛创意设计大赛\textbf{全国特等奖}( \textit{http://www.rcccaa.org/} ),2013年8月
%%   \item 中国机器人大赛暨Robocup公开赛(武术擂台赛)全国一等奖,2013年10月
%  \item 第11届北京理工大学“世纪杯”竞赛学生课外科技作品竞赛\textbf{特等奖},2013年8月
%  \item VIS Components for security system, \textit{https://hijiangtao.github.io/ss-vis-component/}
%  \item 个人博客:\textit{https://hijiangtao.github.io/},更多作品见 \textit{https://github.com/hijiangtao}
%%   \item 电视节目"爸爸去哪儿"可视化分析展示, \textit{https://hijiangtao.github.io/variety-show-hot-spot-vis/}
%\end{itemize}

% \section{\faHeartO\ 项目/作品摘要}
% \section{项目/作品摘要}
% \datedline{\textit{An Integrated Version of Security Monitor Vis System}, https://hijiangtao.github.io/ss-vis-component/ An Integrated Version of Security Monitor Vis System}{blabla}
% \datedline{\textit{Dark-Tech}, https://github.com/hijiangtao/dark-tech/ }{}
% \datedline{\textit{融合社交网络数据挖掘的电视节目可视化分析系统}, https://hijiangtao.github.io/variety-show-hot-spot-vis/}{}
% \datedline{\textit{LeetCodeOJ Solutions}, https://github.com/hijiangtao/LeetCodeOJ}{}
% \datedline{\textit{Info-Vis}, https://github.com/ISCAS-VIS/infovis-ucas}{}


\end{document}
